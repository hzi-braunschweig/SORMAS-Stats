% Options for packages loaded elsewhere
\PassOptionsToPackage{unicode}{hyperref}
\PassOptionsToPackage{hyphens}{url}
%
\documentclass[
  8pt,
  ignorenonframetext,
]{beamer}
\usepackage{pgfpages}
\setbeamertemplate{caption}[numbered]
\setbeamertemplate{caption label separator}{: }
\setbeamercolor{caption name}{fg=normal text.fg}
\beamertemplatenavigationsymbolsempty
% Prevent slide breaks in the middle of a paragraph
\widowpenalties 1 10000
\raggedbottom
\setbeamertemplate{part page}{
  \centering
  \begin{beamercolorbox}[sep=16pt,center]{part title}
    \usebeamerfont{part title}\insertpart\par
  \end{beamercolorbox}
}
\setbeamertemplate{section page}{
  \centering
  \begin{beamercolorbox}[sep=12pt,center]{part title}
    \usebeamerfont{section title}\insertsection\par
  \end{beamercolorbox}
}
\setbeamertemplate{subsection page}{
  \centering
  \begin{beamercolorbox}[sep=8pt,center]{part title}
    \usebeamerfont{subsection title}\insertsubsection\par
  \end{beamercolorbox}
}
\AtBeginPart{
  \frame{\partpage}
}
\AtBeginSection{
  \ifbibliography
  \else
    \frame{\sectionpage}
  \fi
}
\AtBeginSubsection{
  \frame{\subsectionpage}
}
\usepackage{lmodern}
\usepackage{amssymb,amsmath}
\usepackage{ifxetex,ifluatex}
\ifnum 0\ifxetex 1\fi\ifluatex 1\fi=0 % if pdftex
  \usepackage[T1]{fontenc}
  \usepackage[utf8]{inputenc}
  \usepackage{textcomp} % provide euro and other symbols
\else % if luatex or xetex
  \usepackage{unicode-math}
  \defaultfontfeatures{Scale=MatchLowercase}
  \defaultfontfeatures[\rmfamily]{Ligatures=TeX,Scale=1}
\fi
% Use upquote if available, for straight quotes in verbatim environments
\IfFileExists{upquote.sty}{\usepackage{upquote}}{}
\IfFileExists{microtype.sty}{% use microtype if available
  \usepackage[]{microtype}
  \UseMicrotypeSet[protrusion]{basicmath} % disable protrusion for tt fonts
}{}
\makeatletter
\@ifundefined{KOMAClassName}{% if non-KOMA class
  \IfFileExists{parskip.sty}{%
    \usepackage{parskip}
  }{% else
    \setlength{\parindent}{0pt}
    \setlength{\parskip}{6pt plus 2pt minus 1pt}}
}{% if KOMA class
  \KOMAoptions{parskip=half}}
\makeatother
\usepackage{xcolor}
\IfFileExists{xurl.sty}{\usepackage{xurl}}{} % add URL line breaks if available
\IfFileExists{bookmark.sty}{\usepackage{bookmark}}{\usepackage{hyperref}}
\hypersetup{
  pdftitle={Data visualization for SORMAS},
  pdfauthor={Stéphane Ghozzi Hemholtz Centre for Infection Research (HZI) stephane.ghozzi@helmholtz-hzi.de },
  hidelinks,
  pdfcreator={LaTeX via pandoc}}
\urlstyle{same} % disable monospaced font for URLs
\newif\ifbibliography
\setlength{\emergencystretch}{3em} % prevent overfull lines
\providecommand{\tightlist}{%
  \setlength{\itemsep}{0pt}\setlength{\parskip}{0pt}}
\setcounter{secnumdepth}{-\maxdimen} % remove section numbering
\usepackage{graphicx}
\usepackage[T1]{fontenc}
\usepackage[USenglish]{babel}
\usepackage{hyperref}
\usepackage{url}
\usepackage{cite}
\usepackage{amssymb}
\usepackage{amsmath}
\usepackage{bm}
\usepackage{tikz}

\setbeamertemplate{navigation symbols}{}
\defbeamertemplate{footline}{title and page number}{%
    \color{gray!95}
    \hspace{2em} \inserttitle \hfill%
    \insertpagenumber\,/\,\insertpresentationendpage\kern2em\vskip2pt%
    \vspace{0.2cm}%
}
\setbeamertemplate{footline}[title and page number]{}
\setbeamertemplate{frametitle}{
    \vspace{0.5cm}
    \insertframetitle
}
\setlength{\fboxsep}{0pt}
\setlength{\fboxrule}{0.1pt}
\useinnertheme{circles}
\usefonttheme{serif}

\definecolor{vegablue}{HTML}{1F77B4}
\definecolor{vegaorange}{HTML}{FF7F0E}
\definecolor{vegagreen}{HTML}{2CA02C}
\setbeamercolor{title}{fg=black}
\setbeamercolor{subtitle}{fg=vegablue}
\setbeamercolor{frametitle}{fg=vegablue}
\setbeamercolor{structure}{fg=black}

\hypersetup{colorlinks=true,linkcolor=vegablue,urlcolor=vegablue,citecolor=vegablue,anchorcolor=vegablue}

\newcommand{\p}{\mathrm{p}}
\newcommand{\x}{\bm{x}}
\ifluatex
  \usepackage{selnolig}  % disable illegal ligatures
\fi

\title{Data visualization for SORMAS}
\subtitle{10 November 2020}
\author{\newline \newline Stéphane Ghozzi \newline \small Hemholtz
Centre for Infection Research (HZI)
\newline \href{mailto:stephane.ghozzi@helmholtz-hzi.de}{\nolinkurl{stephane.ghozzi@helmholtz-hzi.de}}
\normalsize}
\date{}

\begin{document}
\frame{\titlepage}

\begin{frame}{Software}
\protect\hypertarget{software}{}
SORMAS is an infectious-disease surveillance and response tool for
recording:

\begin{itemize}
\tightlist
\item
  \textbf{cases} (persons infected)
\item
  their \textbf{contacts}
\item
  \textbf{events} and their \textbf{participants}
\end{itemize}

\vspace{0.75cm}

open source

deployed in France, Germany, Ghana, Nigeria, Switzerland

HZI + partners: development, training, support

current development focus on \textbf{COVID-19}

\vspace{0.75cm}

\url{https://www.sormas.org/} \newline \url{https://www.sormas-oegd.de/}
\newline \url{https://github.com/hzi-braunschweig/SORMAS-Project}
\end{frame}

\begin{frame}{Data}
\protect\hypertarget{data}{}
High dimensional:

\begin{itemize}
\tightlist
\item
  person (address, phone number, sex, age, occupation, \ldots)
\item
  disease course (symptoms, laboratory tests, \ldots)
\item
  context (setting of event, \ldots)
\item
  workflow of public-health workers (contact in quarantaine, \ldots)
\end{itemize}

\ldots{} and much more!

N.B. strong data privacy

\vspace{0.75cm}

In progress: generation of a \textbf{credible synthetic data set} for

\begin{itemize}
\tightlist
\item
  software testing
\item
  training
\item
  \textbf{analyses, visualizations, collaborations}
\end{itemize}
\end{frame}

\begin{frame}{Visualizations}
\protect\hypertarget{visualizations}{}
\begin{block}{The usual}
\protect\hypertarget{the-usual}{}
\vspace{0.5cm}

\begin{itemize}
\tightlist
\item
  distributions
\item
  time series
\item
  choropleth maps
\item
  \ldots{}
\end{itemize}

\vspace{1cm}
\end{block}

\begin{block}{Indicators}
\protect\hypertarget{indicators}{}
\vspace{0.5cm}

\begin{itemize}
\tightlist
\item
  reproduction number \(R_e\)
\item
  dispersion factor \(K\)
\item
  anomaly detection
\item
  \ldots{}
\end{itemize}
\end{block}
\end{frame}

\begin{frame}
\begin{block}{Networks}
\protect\hypertarget{networks}{}
\vspace{0.5cm}

the most interesting\ldots{} and challenging!

\vspace{0.5cm}

Idea/requirement:

\begin{itemize}
\tightlist
\item
  graphs of \textbf{infections}, \textbf{contacts} and \textbf{event
  participation}
\item
  both \textbf{abstract} and in geographical \textbf{space}
\item
  highlight necessary collaboration between adminstartive units
\end{itemize}

\vspace{0.5cm}

Difficulty:

\begin{itemize}
\tightlist
\item
  quickly difficult to read, e.g.~\textasciitilde{} 13,000 persons and
  events in COVID-19 test data for only 3 counties
\item
  high-dimensionality of relevant information
\end{itemize}

\vspace{0.5cm}

\(\Longrightarrow\) need to hierarchize and navigate information!
\end{block}
\end{frame}

\begin{frame}{Networks: first idea}
\protect\hypertarget{networks-first-idea}{}
\small

In progress, rough sketch\ldots{} still messy!

\vspace{0.25cm}

\begin{enumerate}
\tightlist
\item
  Build graph:
\end{enumerate}

\begin{itemize}
\tightlist
\item
  nodes = persons and events
\item
  edges = infection, contact and participation
\end{itemize}

\vspace{0.25cm}

\begin{enumerate}
\setcounter{enumi}{1}
\tightlist
\item
  Filter nodes:
\end{enumerate}

\begin{itemize}
\tightlist
\item
  by time, e.g.~of reporting
\item
  by \textbf{components} (connected subgraphs) \textasciitilde{}
  clusters
\end{itemize}

\normalsize

\tiny

N.B. COVID19 component sizes skewed: most cases infect no one, a few
larger clusters \normalsize

\vspace{0.25cm}

\small

\begin{enumerate}
\setcounter{enumi}{2}
\tightlist
\item
  Aggregation at regional level:
\end{enumerate}

\begin{itemize}
\tightlist
\item
  highlight \emph{trans-regional} components
\end{itemize}

\vspace{0.25cm}

\begin{enumerate}
\setcounter{enumi}{3}
\tightlist
\item
  Visualization:
\end{enumerate}

\begin{itemize}
\tightlist
\item
  node color = type of person or event
\item
  node size = degree or betweenness (aggregated: number of persons)
\item
  edge type = type relation (aggregated: number of relations)
\item
  \textbf{interactive}
\end{itemize}

\normalsize
\end{frame}

\begin{frame}{Example 1: component with infection chain}
\protect\hypertarget{example-1-component-with-infection-chain}{}
abstract visualization

\begin{center}
\includegraphics[width=0.8\textwidth]{../img/components/14/abstract_graph.pdf}
\end{center}
\end{frame}

\begin{frame}
static visualization in space

\begin{center}
\includegraphics[width=0.8\textwidth]{../img/components/14/geo_graph_map.pdf}
\end{center}
\end{frame}

\begin{frame}
interactive visualization in space

\begin{center}
\includegraphics[width=0.8\textwidth]{../img/components/14/subnetwork_component.png}
\end{center}
\end{frame}

\begin{frame}{Example 2: component with event}
\protect\hypertarget{example-2-component-with-event}{}
abstract visualization

\begin{center}
\includegraphics[width=0.8\textwidth]{../img/components/55/abstract_graph.pdf}
\end{center}
\end{frame}

\begin{frame}
static visualization in space

\begin{center}
\includegraphics[width=0.8\textwidth]{../img/components/55/geo_graph_map.pdf}
\end{center}
\end{frame}

\begin{frame}
interactive visualization in space

\begin{center}
\includegraphics[width=0.8\textwidth]{../img/components/55/subnetwork_component.png}
\end{center}
\end{frame}

\end{document}
